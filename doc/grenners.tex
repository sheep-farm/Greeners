\documentclass[11pt, a4paper]{article}
\usepackage[utf8]{inputenc}
\usepackage{amsmath, amssymb, amsthm}
\usepackage{bm}
\usepackage{geometry}
\usepackage{listings}
\usepackage{xcolor}
\usepackage{hyperref}

\geometry{left=2.5cm, right=2.5cm, top=2.5cm, bottom=2.5cm}

% Definições Matemáticas
\newcommand{\y}{\mathbf{y}}
\newcommand{\x}{\mathbf{x}}
\newcommand{\X}{\mathbf{X}}
\newcommand{\betavec}{\boldsymbol{\beta}}
\newcommand{\epsilonvec}{\boldsymbol{\epsilon}}
\newcommand{\uvec}{\mathbf{u}}
\newcommand{\A}{\mathbf{A}}
\newcommand{\M}{\mathbf{M}}
\newcommand{\I}{\mathbf{I}}
\newcommand{\Sigmavec}{\boldsymbol{\Sigma}}
\newcommand{\Omegavec}{\boldsymbol{\Omega}}

% Configuração de Código
\definecolor{codegreen}{rgb}{0,0.6,0}
\definecolor{codegray}{rgb}{0.5,0.5,0.5}
\definecolor{codepurple}{rgb}{0.58,0,0.82}
\definecolor{backcolour}{rgb}{0.95,0.95,0.92}

\lstdefinestyle{ruststyle}{
    backgroundcolor=\color{backcolour},   
    commentstyle=\color{codegreen},
    keywordstyle=\color{magenta},
    numberstyle=\tiny\color{codegray},
    stringstyle=\color{codepurple},
    basicstyle=\ttfamily\footnotesize,
    breakatwhitespace=false,         
    breaklines=true,                 
    captionpos=b,                    
    keepspaces=true,                 
    numbers=left,                    
    numbersep=5pt,                  
    showspaces=false,                
    showstringspaces=false,
    showtabs=false,                  
    tabsize=2,
    language=C++ % Rust syntax highlighting approximation
}

\lstset{style=ruststyle}

\title{\textbf{Greeners: Mathematical Reference Manual} \\ \large Version 0.1.0}
\author{Documentation \& Technical Reference}
\date{\today}

\begin{document}

\maketitle
\tableofcontents
\newpage

\section{Introduction}
Greeners is an econometrics library implemented in Rust, designed for high-performance numerical computation using LAPACK/BLAS backends. This document outlines the mathematical formulations used in the estimators.

\section{Linear Models (OLS \& GLS)}
The fundamental estimator is Ordinary Least Squares (OLS). Given the model $\y = \X\betavec + \epsilonvec$, the estimator is:
$$ \hat{\betavec}_{OLS} = (\X'\X)^{-1}\X'\y $$
Variance-Covariance matrix estimation supports robust specifications:
\begin{itemize}
    \item \textbf{Homoskedastic:} $\hat{V} = s^2 (\X'\X)^{-1}$
    \item \textbf{White (HC0):} $\hat{V} = (\X'\X)^{-1} (\X' \text{diag}(\hat{u}_i^2) \X) (\X'\X)^{-1}$
    \item \textbf{Newey-West (HAC):} Adjusts for autocorrelation via a kernel-weighted sum of autocovariances.
\end{itemize}

\section{Quantile Regression}
Unlike OLS which estimates the conditional mean, Quantile Regression estimates the conditional quantile $Q_\tau(\y|\X)$. It minimizes the check loss function:
$$ \min_{\betavec} \sum_{i=1}^n \rho_\tau (y_i - \x_i'\betavec) $$
where $\rho_\tau(u) = u(\tau - \mathbb{I}(u < 0))$.

\textbf{Implementation:} Greeners solves this optimization problem using \textbf{Iteratively Reweighted Least Squares (IRLS)}. At each step $k$, weights $w_i = |e_i|^{-1}$ are updated until convergence.

\section{Discrete Choice (Logit \& Probit)}
For binary dependent variables $y_i \in \{0, 1\}$, we model $P(y_i=1|\x_i) = G(\x_i'\betavec)$.
\begin{itemize}
    \item \textbf{Logit:} $G(z) = \frac{e^z}{1+e^z}$ (Logistic CDF)
    \item \textbf{Probit:} $G(z) = \Phi(z)$ (Standard Normal CDF)
\end{itemize}
Estimation is performed via \textbf{Maximum Likelihood Estimation (MLE)} using the Newton-Raphson algorithm with analytic gradients and Hessians.

\section{Panel Data Models}
\subsection{Fixed Effects (Within Estimator)}
Designed to eliminate time-invariant unobserved heterogeneity $\alpha_i$. Greeners applies the within transformation:
$$ (y_{it} - \bar{y}_i) = (\x_{it} - \bar{\x}_i)'\betavec + (u_{it} - \bar{u}_i) $$
This is estimated via OLS on the demeaned data. Degrees of freedom are corrected as $N - K - (N_{entities}-1)$.

\subsection{Random Effects (Swamy-Arora)}
Assumes $\alpha_i$ is uncorrelated with $\x_{it}$. It is a GLS estimator where data is quasi-demeaned by $\theta$:
$$ \theta = 1 - \sqrt{\frac{\sigma_u^2}{\sigma_u^2 + T\sigma_\alpha^2}} $$
The transformed model is $(y_{it} - \theta \bar{y}_i) = (\x_{it} - \theta \bar{\x}_i)'\betavec + \dots$

\subsection{Dynamic Panel (Arellano-Bond)}
For models with lagged dependent variables ($y_{i,t-1}$), the Fixed Effects estimator is inconsistent (Nickell Bias). Greeners implements the \textbf{Difference GMM} estimator:
\begin{enumerate}
    \item Take first differences to remove $\alpha_i$: $\Delta y_{it} = \rho \Delta y_{i,t-1} + \Delta \betavec \x_{it} + \Delta u_{it}$
    \item Use lagged levels $y_{i, t-2}$ as instruments for $\Delta y_{i,t-1}$.
\end{enumerate}

\subsection{Panel Threshold (Hansen, 1999)}
Captures non-linear regime switching based on a threshold variable $q_{it}$:
$$ y_{it} = \mu_i + \betavec_1' \x_{it} I(q_{it} \le \gamma) + \betavec_2' \x_{it} I(q_{it} > \gamma) + e_{it} $$
Greeners estimates $\gamma$ via a grid search that minimizes the sum of squared residuals (SSR).

\section{Vector Autoregression (VAR) \& VECM}
\subsection{VAR}
A multivariate system $\y_t = \A_1 \y_{t-1} + \dots + \A_p \y_{t-p} + \uvec_t$.
Greeners provides:
\begin{itemize}
    \item OLS/SUR estimation.
    \item AIC/BIC selection.
    \item \textbf{Impulse Response Functions (IRF):} Computed via Cholesky decomposition of $\Sigma_u$ to identify structural shocks.
\end{itemize}

\subsection{VECM (Johansen Procedure)}
For cointegrated I(1) variables. The model is:
$$ \Delta \y_t = \bm{\alpha}\bm{\beta}' \y_{t-1} + \sum_{i=1}^{p-1} \bm{\Gamma}_i \Delta \y_{t-i} + \uvec_t $$
Estimation involves solving the generalized eigenvalue problem:
$$ |\lambda \mathbf{S}_{11} - \mathbf{S}_{10} \mathbf{S}_{00}^{-1} \mathbf{S}_{01}| = 0 $$
where $\mathbf{S}_{ij}$ are moment matrices of residuals from auxiliary regressions. The eigenvectors associated with the largest eigenvalues form the cointegration matrix $\bm{\beta}$.

\section{Systems of Equations}
\subsection{SUR (Zellner)}
Efficiently estimates a system of equations with correlated error terms using Feasible Generalized Least Squares (FGLS).

\subsection{3SLS (Three-Stage Least Squares)}
Combines 2SLS (IV) and SUR to handle both endogeneity and cross-equation error correlation.

\end{document}